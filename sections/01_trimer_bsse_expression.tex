\section{BSSE Expression for a Trimer}
\label{sec:expr}

Consider a molecular system composed of monomers \(A\), \(B\), and \(C\).
The total energy can be decomposed into one-, two-, and three-body contributions
following the many-body expansion formalism~\autocite{Valiron1997} :
\begin{align}
  \begin{split}
  E^{\mathrm{tot}}_{ABC} &=
  E_A + E_B + E_C 
  +   \varepsilon^{(2)}_{AB}
  +   \varepsilon^{(2)}_{AC}
  +   \varepsilon^{(2)}_{BC}
  +   \varepsilon^{(3)}_{ABC},\\[0.25cm]
  %--
  E^{\mathrm{tot}}_{ABC} &=
  \sum_{i}E_i + \sum_{i\neq j}\varepsilon_{ij}^{(2)} + \varepsilon^{(3)}_{ABC},
  \label{eq:manybody_trimer}
  \end{split}
\end{align}
where \(i,j\in\{A,B,C\}\) and the ordering \(ij\) is chosen so that \(i<j\)
to avoid double counting.
The two-body terms \(\varepsilon_{ij}^{(2)}\) refer to the interaction energy
between pairs of monomers, while the three-body term
\(\varepsilon_{ABC}^{(3)}\) represents the non-additive energy
that arises only when all three monomers are present simultaneously.
In other words, these terms correspond to the energy required to bring
two or three monomers together, respectively.

\noindent
The standard two-body interaction energy is defined as
\begin{align}
    \Delta E_{\mathrm{int}}^{\mathrm{st}}(ij)
    = \varepsilon_{ij}^{(2)}
    = E_{ij}^{\mathrm{tot}} - (E_i + E_j),
    \label{eq:equation_2}
\end{align}
and, analogously, the standard three-body interaction energy is
\begin{align}
    \Delta E_{\mathrm{int}}^{\mathrm{st}}(ijk)
    = \varepsilon_{ijk}^{(3)}
    = E_{ijk}^{\mathrm{tot}} - (E_i + E_j + E_k).
    \label{eq:st_three_body}
\end{align}

\noindent
By combining Eqs.~(\ref{eq:st_three_body}) and (\ref{eq:manybody_trimer}),
the total interaction energy of the system formed by monomers \(A\), \(B\), and \(C\)
can be expressed in two equivalent ways:
\begin{align}
      \Delta E_{\mathrm{int}}^{\mathrm{st}}
      &= E^{\mathrm{tot}}_{ABC} - \sum_{i}E_i,
      \label{eq:three_body_inter_1}\\[0.25cm]
      %--
      \Delta E_{\mathrm{int}}^{\mathrm{st}}
      &= 
          \varepsilon^{(2)}_{AB}
      +   \varepsilon^{(2)}_{AC}
      +   \varepsilon^{(2)}_{BC}
      +   \varepsilon^{(3)}_{ABC}.
      \label{eq:three_body_inte_2}
\end{align}

\noindent
Equation~(\ref{eq:three_body_inter_1}) is typically used when the total
cluster energy \(E^{\mathrm{tot}}_{ABC}\) and the monomer energies
\(E_A\), \(E_B\), and \(E_C\) are known from direct computation.
Equation~(\ref{eq:three_body_inte_2}), in contrast,
is more convenient when the two-body and three-body
interaction energies are determined separately, for example,
from potential energy surface calculations or incremental schemes.
%--
Thus, the interaction energy of the trimer system can be evaluated
either from the total energy of the cluster and its monomers,
or from the sum of the pairwise and three-body interaction contributions. In theory, if the ``exact'' energies of the monomers and the cluster were available, the interaction energy of a two-particle system could be determined exactly using Eq.~(\ref{eq:equation_2}).
In practice, however, this is not possible because all quantities are obtained through quantum-chemical approximations that depend on both the
\emph{level of theory}---such as Hartree--Fock (HF),
Møller--Plesset perturbation theory (MP2),
coupled-cluster theory with single, double, and perturbative triple excitations [CCSD(T)],
or density functional theory (DFT)---and the chosen \emph{basis set}
used to represent the molecular orbitals.


The fundamental problem that arises is one of \textbf{basis-set inconsistency}.
Within the framework of the variational principle and the
Roothaan--Hall approximation~\autocite{Mayer2003,Szabo2012}, molecular orbitals are expressed as
linear combinations of a finite set of \emph{basis functions}
(often referred to as atomic orbitals).
In this context, the total basis set of the supermolecule
is larger than that of the individual monomers:
the supermolecule basis consists of all basis functions
associated with every monomer in the system,
whereas each isolated monomer is described only by its own
subset of basis functions.
Consequently, the total energy of the cluster is evaluated in a
more complete variational space than that of the separated monomers.


If we compute the standard interaction energy of the \(AB\) cluster
under these conditions, we obtain
\begin{equation}
    \Delta E_{\mathrm{int}}^{\mathrm{st}}
      = E^{\mathrm{tot}}_{AB}(AB)
        - \big[E_{A}(A) + E_{B}(B)\big],
    \label{eq:st_basinconsistency}
\end{equation}
where the notation in parentheses explicitly denotes
the basis set employed for each subsystem.
Because the basis used for the cluster \((AB)\) is more extensive
than that used for the isolated monomers \((A)\) and \((B)\),
the total energy of the dimer is artificially stabilized.
As a result, the corresponding interaction energy
is spuriously lowered,
leading to what is known as the
\emph{basis set superposition error} (BSSE).

To correct for this basis-set inconsistency,
Boys and Bernardi~\cite{Boys1970} introduced the
\textit{counterpoise (CP) correction} procedure.
The method was originally formulated for a two-particle system (\(AB\)),
in which the total energy of the dimer \(AB\)
is evaluated in the full dimer basis \((AB)\),
and each monomer energy is recomputed in the same basis
while the partner monomer is replaced by \emph{ghost} orbitals
(i.e., basis functions without nuclei or electrons).
The counterpoise-corrected interaction energy is then defined as
\begin{equation}
    \Delta E_{\mathrm{int}}^{\mathrm{CP}}
      = E^{\mathrm{tot}}_{AB}(AB)
        - \big[E_{A}(AB) + E_{B}(AB)\big],
    \label{eq:CP_correction_two_body}
\end{equation}
%--
\noindent
where \(E_A(AB)\) and \(E_B(AB)\)
represent the monomer energies calculated in the presence of
the other monomer’s basis set.

%--
\clearpage
The BSSE quantifies how much the standard interaction energy is
\textit{contaminated} by basis-set inconsistencies.
A practical way to evaluate this contamination is to compute
the difference between the standard (uncorrected) interaction energy
and the counterpoise-corrected one\autocite{Duijneveldt1994} :
\begin{equation}
\label{eq:BSSE_definition_two_body}
\mathrm{BSSE}
= \Delta E_{\mathrm{int}}^{\mathrm{st}} - \Delta E_{\mathrm{int}}^{\mathrm{CP}}
= \big[E_{A}(AB)+E_{B}(AB)\big] - \big[E_{A}(A)+E_{B}(B)\big].
\end{equation}

\noindent
Since interaction energies for bound systems are typically negative (attractive), a \emph{negative} BSSE indicates that the uncorrected interaction energy, $\Delta E_{\mathrm{int}}^{\mathrm{st}}$, is \textit{too attractive}. After applying the counterpoise correction, the resulting interaction energy $\Delta E_{\mathrm{int}}^{\mathrm{CP}}$ is therefore \textbf{less negative}, and thus
\[
\mathrm{BSSE} < 0.
\]
This negative value represents the \textbf{fictitious stabilization}—the artificial energy lowering that arises purely from basis-set imbalance. Hence, the difference above quantifies how much of the apparent overbinding is due to BSSE.


One of the first extensions of the counterpoise (CP) correction to trimers was given by White and Davidson in their study of hydrogen-bonded ice\autocite{White1990}, and later generalized by Valiron and Mayer\autocite{Valiron1997}. In this framework, each two-body interaction energy $\varepsilon_{ij}^{(2)}$ ($i,j\in\{A,B,C\}$) is evaluated with CP as in Eq.~(\ref{eq:equation_2}) and assembled in Eq.~(\ref{eq:three_body_inte_2}). The remaining term,
$\varepsilon_{ijk}^{(3)}$, is the \emph{pure three-body} contribution. It is obtained from the many-body expansion for the trimer, Eq.~(\ref{eq:manybody_trimer}), with CP applied consistently to all monomer and dimer terms:
{\small
\begin{equation}
\label{eq:three_body_cp}
\begin{split}
\varepsilon_{ABC}^{(3)}(ABC)
= E^{\mathrm{tot}}_{ABC}(ABC)
- \Big[
  E_{A}(ABC) + E_{B}(ABC) + E_{C}(ABC)
  \\
  \qquad\qquad
  +\, \varepsilon_{AB}^{(2)}(ABC)
  +\, \varepsilon_{AC}^{(2)}(ABC)
  +\, \varepsilon_{BC}^{(2)}(ABC)
\Big].
\end{split}
\end{equation}
}
From here the (CP) interaction energy for the trimer can be written as
\begin{align}
    \begin{split}
        \Delta E_{\text{int}}^{\text{CP}} =
        \varepsilon_{AB}^{(2)}(AB) + \varepsilon_{AC}^{(2)}(AC) + \varepsilon_{BC}^{(2)}(BC) + \varepsilon_{ABC}^{(3)}(ABC)
    \end{split}
\end{align}
which can be arranged into the following 16-term expression by considering each contribution:
{\small
\begin{align}
    \begin{split}
        \Delta E_{\text{int}}^{\text{CP}} &=
        E_{AB}(AB) - \big[E_{A}(AB)+E_{B}(AB)\big] + E_{AC}(AC) -\big[E_{A}(AC)+E_{C}(AC)\big]\\
        &+  E_{BC}(BC)
        -\big[E_{B}(BC)+E_{C}(BC)\big]
        %--
        + \mathcal{E}_{ABC}
        \label{eq:equation-11}
    \end{split}
\end{align}
}
where 
\[
\mathcal{E}_{ABC} =  E_{ABC}(ABC) - \big[E_{AB}(ABC) + E_{AC}(ABC) + E_{BC}(ABC) - \big[E_{A}(ABC) + E_{B}(ABC)+E_{C}(ABC)\big]\big]
\]
%--
\clearpage
Now, the standard interaction energy  for the trimer is given by
\begin{equation}
    \Delta E_{\text{int}}^{\text{st}} = E_{ABC}(ABC) - \big[ 
    E_{A}(A) + E_{B}(B) + E_{C}(C)\big]
    \label{eq:equation-12}
\end{equation}
and from here the BSSE  is given by
\small{
\begin{align}
    \begin{split}
    \text{BSSE} &= \Delta E_{\mathrm{int}}^{\mathrm{st}} - \Delta E_{\mathrm{int}}^{\mathrm{CP}}\\[0.25cm]
    &=\Big[[E_{A}(AB) + E_{A}(AC)] -  [E_{A}(ABC) + E_{A}(A)]\Big]\\[0.25cm]
    %--
    &+\Big[[E_{B}(AB) + E_{B}(BC)] -  [E_{B}(ABC) + E_{B}(B)]\Big]\\[0.25cm]
    %--
    &+\Big[E_{C}(AC) + E_{C}(BC)] -  [E_{C}(ABC) + E_{C}(C)]\Big]\\[0.25cm]
    %--
    &+\Big[[E_{AB}(ABC) + E_{AC}(ABC)]-[E_{AB}(AB)+E_{AC}(AC)]\Big]
    %--
    + \Big[E_{BC}(ABC) - E_{BC}(BC)\Big]
    \end{split}
    \label{eq:equation-13}
\end{align}
}
\noindent





%\begin{align}
%    \begin{split}
%        \Delta E_{\mathrm{int}}^{\mathrm{CP}} =
        
%    \end{split}
%\end{align}






