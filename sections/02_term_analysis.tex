\section{BSSE for the Trimer System: Subset Representation}
\label{sec:bsse_trimer}

In the previous section we saw that the BSSE
[Eq.~(\ref{eq:equation-13})]
for a trimer depends on the monomer and dimer energies
evaluated in different basis sets.
If we define the set
\(\mathcal{B} = \{A,B,C\}\)
to represent the basis sets associated with each monomer,
we can describe the hierarchy of energy evaluations
in terms of the subsets of \(\mathcal{B}\).

\noindent
The \textbf{power set} of \(\mathcal{B}\),
denoted \(\mathcal{P}_{\mathcal{B}}\),
is the set of all possible subsets of \(\mathcal{B}\):
\[
\mathcal{P}_{\mathcal{B}}
   = \bigl\{
      \emptyset,
      \{A\},\{B\},\{C\},
      \{A,B\},\{A,C\},\{B,C\},
      \{A,B,C\}
     \bigr\}.
\]
The power set contains every possible combination of the basis sets,
including the empty set,
and its cardinality is given by
\[
|\mathcal{P}_{\mathcal{B}}| = 2^{|\mathcal{B}|} = 2^3 = 8.
\]

\noindent
We are interested in the subsets that contain a given element. So, for any element \(X \in \mathcal{B}\),
we can define the collection of all subsets that contain \(X\) as
\[
\mathcal{P}_{\mathcal{B}}(X)
   = \{\, S \subseteq \mathcal{B} \mid X \in S \,\}.
\]
For instance,
\[
\mathcal{P}_{\mathcal{B}}(A)
   = \{\{A\}, \{A,B\}, \{A,C\}, \{A,B,C\}\}.
\]
%--
\clearpage
This shows that each monomer (here \(A\)) appears in four different subsets of \(\mathcal{B}\). Then, every subset \(S\) of \(\mathcal{B}\) that contains \(A\) can be written uniquely as
\[
S = \{A\} \cup T,
\]
where \(T\) is any subset of \(\mathcal{B}\setminus\{A\}\).
Hence,
\[
\boxed{
\mathcal{P}_{\mathcal{B}}(A)
   = \bigl\{\, \{A\}\cup T \;\big|\;
     T\subseteq\mathcal{B}\setminus\{A\}\,\bigr\}.
}
\]
Now, we can see that there are \(2^{|\mathcal{B}|-1}\) possible subsets
\(T\subseteq\mathcal{B}\setminus\{A\}\),
it follows that
\[
|\mathcal{P}_{\mathcal{B}}(A)| = 2^{|\mathcal{B}|-1}.
\]
For a trimer (\(|\mathcal{B}|=3\)),
we have \( |\mathcal{P}_{\mathcal{B}}(A)| = 2^{2} = 4\),
corresponding exactly to the subsets
\(\{A\}\), \(\{A,B\}\), \(\{A,C\}\), and \(\{A,B,C\}\).

\noindent
In an analogous way, we can determine all subsets of
\(\mathcal{B}\) that contain a specific \emph{pair} of elements,
say \(A\) and \(B\).
These subsets correspond to all basis combinations
that simultaneously include both \(A\) and \(B\),
which are relevant for the evaluation of dimer energies.

\noindent
We define this collection as
\[
\mathcal{P}_{\mathcal{B}}(A,B)
   = \{\, S \subseteq \mathcal{B} \mid \{A,B\} \subseteq S \,\}.
\]
For our trimer example,
\[
\mathcal{P}_{\mathcal{B}}(A,B)
   = \{\{A,B\}, \{A,B,C\}\}.
\]
Each dimer therefore appears in exactly two subsets of \(\mathcal{B}\):
its own dimer basis and the full trimer basis. Following the same reasoning as before,
every subset \(S\) that contains both \(A\) and \(B\)
can be written uniquely as
\[
S = \{A,B\} \cup T,
\]
where \(T\) is any subset of the remaining elements,
\(\mathcal{B}\setminus\{A,B\}\).
Hence,
\[
\boxed{
\mathcal{P}_{\mathcal{B}}(A,B)
   = \bigl\{\, \{A,B\} \cup T
     \;\big|\;
     T\subseteq\mathcal{B}\setminus\{A,B\}\,\bigr\}.
}
\]
Because there are \(2^{|\mathcal{B}|-2}\) possible subsets
\(T\subseteq\mathcal{B}\setminus\{A,B\}\),
the number of subsets containing both \(A\) and \(B\) is
\[
|\mathcal{P}_{\mathcal{B}}(A,B)| = 2^{|\mathcal{B}|-2}.
\]
For the trimer (\(|\mathcal{B}|=3\)),
we obtain \( |\mathcal{P}_{\mathcal{B}}(A,B)| = 2^{1} = 2\),
corresponding to the subsets
\(\{A,B\}\) and \(\{A,B,C\}\),
as expected.

\noindent
More generally, for any \(k\)-tuple of elements
\(\{X_1, X_2, \ldots, X_k\}\subseteq\mathcal{B}\),
the collection of all subsets of \(\mathcal{B}\) containing these \(k\) elements
is given by
%--
\clearpage
\[
\boxed{
\mathcal{P}_{\mathcal{B}}(X_1,X_2,\ldots,X_k)
   = \bigl\{\, \{X_1,X_2,\ldots,X_k\} \cup T
     \;\big|\;
     T\subseteq\mathcal{B}\setminus\{X_1,X_2,\ldots,X_k\}\,\bigr\},
}
\]
with cardinality
\begin{equation}
|\mathcal{P}_{\mathcal{B}}(X_1,X_2,\ldots,X_k)|
   = 2^{|\mathcal{B}|-k}.
\label{eq:equation-14}
\end{equation}
This relation shows that:
\begin{itemize}[topsep=-0.25cm]
    \item each monomer (\(k=1\)) appears in \(2^{|\mathcal{B}|-1}\) subsets,
    \item each dimer (\(k=2\)) appears in \(2^{|\mathcal{B}|-2}\) subsets,
    \item each trimer (\(k=3\)) appears in \(2^{|\mathcal{B}|-3}\) subsets, and so on.
\end{itemize}
This hierarchical pattern is precisely what underlies
the structure of the BSSE corrections
in the Valiron--Mayer formulation of the many-body counterpoise method.



\noindent
This formalism provides a clear combinatorial interpretation
of the BSSE hierarchy:
each monomer must be evaluated in all subsets
that contain it,
while each dimer appears only in the subsets
that contain both of its monomers.
For the trimer case (\(N=3\)),
each monomer energy therefore involves
four basis sets
(its own, two dimer bases, and the trimer basis),
whereas each dimer energy appears in two
(its own and the trimer basis).
This subset structure offers a systematic and general framework
for expressing and counting the hierarchical counterpoise corrections
in any \(N\)-body cluster.

